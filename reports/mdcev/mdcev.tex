\documentclass[12pt,a4paper]{article}

\usepackage[T1]{fontenc}
\usepackage[latin9]{inputenc}
\usepackage{epsfig}
\usepackage{michel}

\usepackage[dcucite,abbr]{harvard}
\harvardparenthesis{none}
\harvardyearparenthesis{round}
%\newcommand{\citeasnoun}[1]{\cite{#1}}

\usepackage{listings}
\usepackage{inconsolata}
\lstset{language=Python}
\lstset{numbers=none, basicstyle=\ttfamily\footnotesize,
  numberstyle=\tiny,keywordstyle=\color{blue},stringstyle=\ttfamily,showstringspaces=false}
\lstset{backgroundcolor=\color[rgb]{0.95 0.95 0.95}}
\lstdefinestyle{numbers}{numbers=left, stepnumber=1,
  numberstyle=\tiny,basicstyle=\footnotesize, numbersep=10pt}
\lstdefinestyle{nonumbers}{numbers=none}
\lstset{
  breaklines=true,
  breakatwhitespace=true,
}

%%%%%%%%%%%%%%%%%%%%%%%%%%%%%% User specified LaTeX commands.
\usepackage{euler,times}

%\usepackage{euler,beton}

\newcommand{\footnoteremember}[2]{
\footnote{#2}
  \newcounter{#1}
  \setcounter{#1}{\value{footnote}}
} 
\newcommand{\footnoterecall}[1]{
\footnotemark[\value{#1}]
} 

\renewcommand{\L}{\mathcal{L}}

%\usepackage[french]{babel}

\title{
  \vspace{-3cm}
  \epsfig{figure=transp-or.eps,height=2cm}
  \hfill
  \epsfig{figure=epfl,height=1.5cm}   \\*[-0.5cm]
  \mbox{}\hrulefill\mbox{} \\*[3cm] Estimating the MDCEV model with Biogeme}
\author{Michel Bierlaire \and Mengyi Wang}
\date{\today}


\begin{document}


\begin{titlepage}
\pagestyle{empty}

\maketitle
\vspace{2cm}

%%%%%%%%%%%%%%%%%%
%%%%%%%%%%%%%%%%%
%The report number is coded as YYMMDD where YY is the year, MM the
%month and DD the day. The number should be unique. If, by any chance,
%two reports are produced the same, assign to one of then the number
%corresponding to the date of the day after.
%When a manuscript is finished, produce two versions. One with the
%report number, and publish it as a technical report on our website,
%one without the report number, and submit it to a journal.
%In this case, use the template of the journal, or simply comment out the next lines.

\begin{center}
  DRAFT
\small Report TRANSP-OR 24xxxx \\ Transport and Mobility Laboratory \\ School of Architecture, Civil and Environmental Engineering \\ Ecole Polytechnique F\'ed\'erale de Lausanne \\ \verb+transp-or.epfl.ch+
\end{center}


\end{titlepage}


\section*{Abstract}

The abstract...
\section{The MDCEV model}

The multiple discrete-continuous extreme value model (MDCEV) is a
choice model where the choice of multiple alternatives can occur
simultaneously. It has been introduced by \citeasnoun{BHAT2005679},
building on the Karush-Kuhn-Tucker multiple discrete-continuous economic
model proposed by \citeasnoun{wales1983estimation}. In this document, we introduce a generalization of the model, where the derivation is performed for a generic utility function. This is motivated by the need to obtain an implementation that is easily extendible to new models in the future.

Consider an individual, denoted as $n$, who is presented with a
distinct set of items, represented as $\mathcal{C}_n$, containing
$J_n$ items in total. Given a total budget of $E_n$, this individual
decides on purchasing a quantity $y_{in}$ of each item. This decision
must verify the following budget constraint:
\[
\sum_{i \in \mathcal{C}_n} e_{in} = \sum_{i \in \mathcal{C}_n} p_{in} y_{in} = E_n,
\]
where:
\begin{itemize}
    \item $p_{in} > 0$ is the price per unit of item $i$ for the individual $n$.
    \item $e_{in} = p_{in} y_{in}$ represents the total expenditure by individual $n$ on item $i$.
\end{itemize}
We assume that the entire budget is spent, and that at least one item is consumed.

Each item $i$ is associated with a utility $U_{in}(y)$, that depends
on the consumed quantity $y$.  We identify two
properties that the utility functions must have. Both are relatively
weak conditions. The first one is related to the optimality
conditions, and are verified if the functions are concave, or
quasi-concave. The second one is related to the specification of the
error terms, needed for the derivation of the econometric model.
It is assumed that there exists an order preserving function $\phi$ such that the random utility can be expressed in additive form, that is such that
\begin{equation}
\phi\left(\frac{\partial U_{in}}{\partial e_i}\right) = V_{in} + \varepsilon_{in},
\end{equation}
where $V_{in}$ is the deterministic part, and $\varepsilon_{in}$ is a random disturbance.
On
top of this, the specification of the utility functions should be
associated with specific behavioral assumptions. This aspect is not discussed in this document, in order to keep the model as general as possible. 

The purchase decisions $y_{1n}, \ldots, y_{J_n, n}$ are assumed to be the solution of the following optimization problem:
\begin{equation}
\max_{y} \sum_{i \in \C_n} U_{in}(y_i)
\end{equation}
subject to
\begin{align}
  \sum_{i \in \mathcal{C}_n} p_{in} y_{i} &= E_n,\\
 y_{i} &\geq 0,
\end{align}
We refer the reader to Appendix~\ref{eq:derivation}, where the model is derived from first principles.

\section{Model specifications}

Biogeme implements the following specifications for the utility functionx.

\subsection{Translated utility function}
In the context of a time use model, \citeasnoun{BHAT2005679} uses the translated utility function introduced by \citeasnoun[Eq. (1)]{Jaehwan-Kim:2002aa}:
\begin{equation}
  \label{eq:orig_u}
U_{in}(e_i) = \exp(\beta^T x_{in} + \varepsilon_{in})(e_i + \gamma_i)^{\alpha_i},
\end{equation}
where $0 < \alpha_i < 1$ and $\gamma_i > 0$ are parameters to be
estimated. Note that there is no price involved in this specification,
as it models time and not goods. It is equivalent to set $p_i=1$, for all $i$ in the model
described above. Note also that,
\citeasnoun{Jaehwan-Kim:2002aa} and \citeasnoun{BHAT2005679} impose
the following restriction on $\alpha_i$: $0 < \alpha_i \leq
1$. However, in the context of this implementation, $\alpha_i=1$ would
create a singularity.

We can calculate
\begin{equation}
\frac{\partial U_{in}}{\partial e_i} = \exp(\beta^T x_{in} + \varepsilon_{in})\alpha_i(e_i + \gamma)^{\alpha_i-1}.
\end{equation}
In this context, we use the logarithm as the order preserving function to obtain the following specification:
\[
\phi\left(\frac{\partial U_{in}}{\partial e_i}\right) =\beta^T x_{in} + \varepsilon_{in} + \ln \alpha_i + (\alpha_i-1) \ln (e_i + \gamma_i),
\]
so that
\[
V_{in} = \beta^T x_{in} + \ln \alpha_i + (\alpha_i-1) \ln (e_i + \gamma_i),
\]
and
\[
c_{in} = -\frac{\partial V_{in}}{\partial e_i} = \frac{1-\alpha_i}{e_i + \gamma_i}.
\]
In Biogeme, this model is referred to as
\lstinline@translated@.


\section{Generalized translated utility function}

\citeasnoun{BHAT2008274} generalizes the above formulation and introduces the following specification, where the utility functions $U_{in}$ are defined as
\begin{equation}
  \label{eq:generalized_utility_outside}
U_{1n} =\exp(\beta^T x_{1n} + \varepsilon_{1n}) \frac{1}{\alpha_1} \left(\frac{e_1}{p_1}\right)^{\alpha_1},
\end{equation}
for the ``outside good'' that is always consumed and, for $i > 1$,
\begin{equation}
  \label{eq:generalized_utility}
U_{in} =\exp(\beta^T x_{in} + \varepsilon_{in}) \frac{\gamma_i}{\alpha_i} \left[\left(\frac{e_i}{p_i \gamma_i}+1\right)^{\alpha_i}-1\right],
\end{equation}
where $\beta$, $0 < \alpha_i < 1$ and $\gamma_i > 0$ are parameters to be
estimated. We can calculate
\begin{equation}
\frac{\partial U_{1n}}{\partial e_1} =\exp(\beta^T x_{1n} + \varepsilon_{1n}) \frac{1}{p_1} \left(\frac{e_1}{p_1}\right)^{\alpha_1-1},
\end{equation}
and
\begin{equation}
  \frac{\partial U_{in}}{\partial e_i} = \exp(\beta^T x_{in} + \varepsilon_{in}) \frac{1}{p_i} \left(\frac{e_i}{p_i \gamma_i}+1\right)^{\alpha_i-1}.
\end{equation}
We use again the logarithm as the order preserving function to obtain the following specification:
\[
\phi\left(\frac{\partial U_{1n}}{\partial e_1}\right) = \beta^T x_{1n} + \varepsilon_{1n} + (\alpha_1-1) \ln e_1 - \alpha_1 \ln p_1,
\]
and
\[
\phi\left(\frac{\partial U_{in}}{\partial e_i}\right) = \beta^T x_{in} + \varepsilon_{in} - \ln p_i + (\alpha_i-1) \ln \left(\frac{e_i}{p_i \gamma_i}+1\right),
\]
so that
\[
V_{1n} = \beta^T x_{1n} + (\alpha_1-1) \ln e_1 - \alpha_1 \ln p_1,
\]
and
\[
V_{in} = \beta^T x_{in} - \ln p_i + (\alpha_i-1) \ln \left(\frac{e_i}{p_i \gamma_i}+1\right).
\]
Finally, we have
\[
c_{1n} = -\frac{\partial V_{1n}}{\partial e_1} = \frac{1-\alpha_1}{e_1},
\]
and
\[
c_{in} = -\frac{\partial V_{in}}{\partial e_i} =\frac{1-\alpha_i}{e_i + p_i \gamma_i}.
\]
In Biogeme, this model is referred to as \lstinline@generalized@.

\subsection{The $\gamma$-profile}
If $\alpha_i \to 0$, \req{eq:generalized_utility} collapses to the linear expenditure system (LES) form, defined as follows:
\begin{equation}
  \label{eq:u_les_outside}
U_{1n}(e_{1})= \exp(\beta^T x_{1n} + \varepsilon_{1n}) \ln \left(\frac{e_1}{p_1} \right),
\end{equation}
and
\begin{equation}
  \label{eq:u_les}
U_{in}(e_{i})= \exp(\beta^T x_{in} + \varepsilon_{in}) \gamma_i \ln \left(\frac{e_i}{p_i \gamma_i}+1 \right),
\end{equation}
where $\beta$, $\gamma_i > 0$ is a  parameter to be
estimated. We can calculate
\[
\frac{\partial U_{1n}}{\partial e_1} =  \exp(\beta^T x_{1n} + \varepsilon_{1n}) \frac{1}{e_1},
\]
and
\[
\frac{\partial U_{in}}{\partial e_i} =  \exp(\beta^T x_{in} + \varepsilon_{in}) \frac{\gamma_i}{e_i + p_i \gamma_i}.
\]
Taking again the logarithm, we have
\[
\phi\left(\frac{\partial U_{1n}}{\partial e_1}\right) =\beta^T x_{1n} + \varepsilon_{1n} - \ln e_1,
\]
and
\[
\phi\left(\frac{\partial U_{in}}{\partial e_i}\right) =\beta^T x_{in} + \varepsilon_{in}) + \ln \gamma_i - \ln(e_i + p_i \gamma_i),
\]
so that
\[
V_{1n} = \beta^T x_{1n}  - \ln e_i,
\]
and
\[
V_{in} = \beta^T x_{in} + \ln \gamma_i - \ln(e_i + p_i \gamma_i).
\]
Finally, we have
\[
c_{1n} = \frac{1}{e_1},
\]
and
\[
c_{in} = \frac{1}{e_i + p_i \gamma_i}.
\]
In Biogeme, this model is referred to as \lstinline@gamma_profile@.

\subsection{The non-monotonic model}

\citeasnoun{Wang:2023aa} introduce the following specification in order to accommodate non-monotonic preferences, motivated in the context of time consumption. With our notations, the specification is
\[
U_{1n}(e_1) =  \frac{1}{\alpha_1}e^{\beta^T x_{in}}\left(\frac{e_1}{p_1}\right)^{\alpha_1}+(\gamma^T z_{in} + \varepsilon_{in}) e_i.
\]
for the outside good, and
\[
U_{in}(e_i) = \frac{\gamma_i}{\alpha_i} e^{\beta^T x_{in}}\left[\left(\frac{e_i}{p_i \gamma_i}+1\right)^{\alpha_i}-1\right]+(\gamma^T z_{in} + \varepsilon_{in}) e_i.
\]
where $\beta$, $\gamma$, $0 < \alpha_i < 1$, and $\gamma_i > 0$  are parameters to be
estimated. In the original paper, the model has been derived using $p_i=1$, for all $i$.
We have
\[
\frac{\partial U_{1n}}{\partial e_1} = \frac{1}{p_1} e^{\beta^T x_{1n}}  \left(\frac{e_1}{p_1}\right)^{\alpha_1-1} + \gamma^T z_{1n} + \varepsilon_{1n},
\]
and
\[
\frac{\partial U_{in}}{\partial e_i} = \frac{1}{p_i} e^{\beta^T x_{in}}  \left(\frac{e_i}{p_i \gamma_i}+1\right)^{\alpha_i-1} + \gamma^T z_{in} + \varepsilon_{in}.
\]
Here, we do not need to apply an order preserving function, so that
\[
V_{1n} = \frac{1}{p_1} e^{\beta^T x_{1n}}  \left(\frac{e_1}{p_1}\right)^{\alpha_1-1} + \gamma^T z_{1n}.
\]
and
\[
V_{in} = \frac{1}{p_i} e^{\beta^T x_{in}}  \left(\frac{e_i}{p_i \gamma_i}+1\right)^{\alpha_i-1} + \gamma^T z_{in}.
\]
We have
\[
c_{1n} = -\frac{\partial V_{1n}}{\partial e_1} =  \frac{1}{p^2_1} e^{\beta^T x_{1n}} (1-\alpha_1)\left(\frac{e_1}{p_1}\right)^{\alpha_1-2},
\]
and
\[
c_{in} = -\frac{\partial V_{in}}{\partial e_i} =  \frac{1}{p^2_i} e^{\beta^T x_{in}} \frac{1-\alpha_i}{\gamma_i}\left(\frac{e_i}{p_i \gamma_i}+1\right)^{\alpha_i-2}.
\]
For the outside good, we have
\[
\ln c_{1n} =  \beta^T x_{1n} + \ln (1-\alpha_1) +(\alpha_i-2) \ln e_1 - \alpha_1 \ln p_1,
\]
and
\[
\frac{1}{c_{1n}} =  p^2_1 e^{-\beta^T x_{1n}} \frac{1}{1-\alpha_1}\left(\frac{e_1}{p_1}\right)^{2-\alpha_1}.
\]
For the other goods, we have
\[
\ln c_{in} = -2\ln p_i + \beta^T x_{in} + \ln (1-\alpha_i) -\ln \gamma_i +(\alpha_i-2) \ln\left(\frac{e_i}{p_i \gamma_i}+1\right).
\]
and
\[
\frac{1}{c_{in}} =  p^2_i e^{-\beta^T x_{in}} \frac{\gamma_i}{1-\alpha_i}\left(\frac{e_i}{p_i \gamma_i}+1\right)^{2-\alpha_i}.
\]

In Biogeme, this model is referred to as \lstinline@non_monotonic@.


%\subsection{Outside good}
%
%
%Note that, if there is an outside good (see \cite[Section 5]{BHAT2008274}), label it alternative one, and define $\psi_{in}=\exp(\varepsilon_{in})$, and
%\begin{equation}
%U_{1n}(e_1) = \frac{1}{\alpha_1} e_1^{\alpha_i}.
%\end{equation}

\bibliographystyle{dcu}
\bibliography{transpor}


\appendix
\section{Derivation of the model}
\label{eq:derivation}
To simplify the derivation that follows, we write the
optimization problem as an equivalent minimization problem, in terms
of expenditures instead of quantities:
\begin{equation}
\min_{e} -\sum_{i \in \C_n} U_{in}(e_i)
\end{equation}
subject to
\begin{align}
  \sum_{i \in \mathcal{C}_n} e_i &= E_n, \label{eq:budget_constraint}\\
 e_{i} &\geq 0. \label{eq:non_negativity}
\end{align}

The optimality conditions can be derived from the Lagrangian of the problem. We introduce a Lagrange multiplier $\lambda$ associated with constraint \req{eq:budget_constraint} and a Lagrange multiplier $\mu_i \geq 0$ for each constraint \req{eq:non_negativity}. The Lagrangian is defined as
\begin{equation}
  \L(y;\lambda, \mu) = -\sum_{i \in \C_n} U_{in}(e_i) + \lambda\left(E_n -  \sum_{i \in \mathcal{C}_n} e_{i}\right) - \sum_{i \in \C_n} \mu_i e_i.
\end{equation}
The first order optimality conditions state that
\begin{equation}
  \label{eq:optimality}
\frac{\partial \L}{\partial e_i} = - \frac{\partial U_{in}}{\partial e_i} - \lambda - \mu_i  = 0, \text{ and } \mu_i e_i = 0, \; \forall i \in \C_n.
\end{equation}
Note that we assume that the second order optimization conditions are also verified. This is the case if the utility functions are concave, for instance.

At least one item is consumed. We assume without loss of generality that it is item 1. Therefore $e_{1n} > 0$ and $\mu_1=0$. Consequently, \req{eq:optimality} can be written
\[
\lambda = - \frac{\partial U_{1n}}{\partial e_1}.
\]
Consider an alternative $j\neq 1$ such that $e_{jn} > 0$. Using the same argument, we can write
\[
\frac{\partial U_{jn}}{\partial e_j} = \frac{\partial U_{1n}}{\partial e_1}.
\]
Consider an alternative $j$ such that $e_{jn} = 0$. In this case, $\mu_j \geq 0$ and
\[
\frac{\partial U_{in}}{\partial e_i}  \leq \frac{\partial U_{1n}}{\partial e_1}.
\]
Note that we can transform the utility functions $U_{in}$ with any order preserving function $F$ without changing the solution of the problem. An order preserving function is a strictly increasing function $F$ of one variable such that $F'(u) > 0$. In that case,
\[
\phi\left(\frac{\partial U_{in}}{\partial e_i}\right) \leq \phi\left(\frac{\partial U_{1n}}{\partial e_1}\right) \Longleftrightarrow
\frac{\partial U_{in}}{\partial e_i}  \leq \frac{\partial U_{1n}}{\partial e_1}.
\]

We now assume that the utility function is a random variable, that can be written in an additive way. More specifically, we assume that there exists an order preserving transform of the utility functions such that
\begin{equation}
\phi\left(\frac{\partial U_{in}}{\partial e_i}\right) = V_{in} + \varepsilon_{in},
\end{equation}
where $V_{in}$ is the deterministic part, and $\varepsilon_{in}$ is a random disturbance. Therefore, the optimality conditions can be written as
\begin{align}
V_{in} + \varepsilon_{in}  &= V_{1n} + \varepsilon_{1n},  & \text{if } e_{in} > 0, \label{eq:non_zero}\\
V_{in} + \varepsilon_{in}  &\leq V_{1n} + \varepsilon_{1n},  & \text{if } e_{in} = 0. \label{eq:zero}
\end{align}
We assume that the utility functions are defined in such a way that
\req{eq:non_zero} defines a bijective relationship between $e_{in}$ and
$\varepsilon_{in}$, for all $i\in\C_n$.

\section{The distribution of expenditures}

We are interested in the distribution of the vector $e_n$, and we have
established that it is a function of the vector of disturbances
$\varepsilon_n$: $e_n = H(\varepsilon_n)$.  Consequently, if we assume
a distribution for $\varepsilon_n$, characterized by a probability
density fonction (pdf) $f_\varepsilon$ and a cumulative distribution
function (CDF) $F_\varepsilon$, we can characterize the distribution of $e_n$.

We start by assuming that $e_{1n}$, the consumed quantity of item 1 is
known and non zero. Consequently, the value of $\varepsilon_{1n}$ is known as
well. In order to derive the pdf evaluated at $e$, we split the vector $e$ into its positive entries $e^+$ and its zero entries $e^0$, alternative 1 being excluded. In an analogous way, we denote $\C^+$ and $\C^0$ the corresponding sets of indices, of size $J^+$ and $J^0$, respectively. 

For each $i \in \C^+$, we can use \req{eq:non_zero}
to define $\varepsilon_n = H^{-1}(e_n)$, where $H^{-1}:\R^{J^+-1}\to\R^{J^+-1}$ is defined as  
\begin{equation}
  \label{eq:change_variable}
H_i^{-1}(e) = \varepsilon_{i+1,n}   =   V_{1n}(e_1) - V_{i+1,n}(e_{i+1}) + \varepsilon_{1n}.
\end{equation}
Therefore, the density function can be decomposed as
\begin{align*}
  f_e(e^+, e^0 | e_1) &= f_e(e^+ | e^0, e_1) \prob(e^0 |e_1) \\
                     &= f_\varepsilon(\varepsilon^+ | \varepsilon^0, \varepsilon_1) \det \left( \frac{\partial H^{-1}}{\partial e}\right)  \prob(e^0 |e_1)
\end{align*}
where
\begin{align*}
  \prob(e^0 |e_1) &= \prob(V_{in} + \varepsilon_{in}  \leq V_{1n} + \varepsilon_{1n},\; \forall i\in\C^0) \\
  &= \prob(\varepsilon_{in}  \leq V_{1n} -V_{in} + \varepsilon_{1n},\; \forall i\in\C^0) \\
  &= F_\varepsilon(\varepsilon_{1n}, 1, \ldots, 1, (V_{1n} -V_{in} + \varepsilon_{1n})_{i\in \C^0}).
\end{align*}
From \req{eq:change_variable}, we can calculate the entries of the Jacobian $\partial H^{-1}/\partial e$. Indeed,
\begin{align*}
  \frac{\partial H_i^{-1}(e)}{\partial e_k} &= \frac{\partial V_{1n}}{\partial e_1}\frac{\partial e_1}{\partial e_k}  & \text{if } k \neq i+1, \\
  &= \frac{\partial V_{1n}}{\partial e_1}\frac{\partial e_1}{\partial e_k} - \frac{\partial V_{i+1,n}}{\partial e_{i+1}}& \text{if } k = i+1.
\end{align*}
From \req{eq:budget_constraint}, we have
\[
e_1 = E - \sum_{j\neq 1} e_j,
\]
so that
\[
\frac{\partial e_1}{\partial e_k} = -1.
\]
Consequently,
\begin{align*}
  \frac{\partial H_i^{-1}(e)}{\partial e_k} &= -\frac{\partial V_{1n}}{\partial e_1}  & \text{if } k \neq i+1, \\
  &= -\frac{\partial V_{1n}}{\partial e_1} - \frac{\partial V_{i+1,n}}{\partial e_{i+1}}& \text{if } k = i+1.
\end{align*}
If we denote
\begin{equation}
c_i =  - \frac{\partial V_{in}}{\partial e_{i}}
\end{equation}
the Jacobian has the following structure:
\[
\partial H^{-1}/\partial e= \left(
\begin{array}{cccc}
  c_1 + c_2 & c_1 & \cdots & c_1 \\
  c_1      & c_1 + c_3 & \cdots & c_1 \\
  &          &   \vdots    \\
  c_1    &  c_1 & \cdots & c_1 + c_{J_n}
\end{array}
\right).
\]
Therefore, the determinant is equal to
\[
\det(\partial H^{-1}/\partial e) = \left(\prod_{i=1}^{J^+} c_i\right)\left(\sum_{i=1}^{J^+}\frac{1}{c_i}\right).
\]
Note that this determinant depends only on the utility function, not on the distribution of the $\varepsilon_n$.

Therefore, the density function of the expenditures is given by
\begin{equation}
  \label{eq:density}
f_e(e^+, e^0) =   \left(\prod_{i=1}^{J^+} c_i\right)\left(\sum_{i=1}^{J^+}\frac{1}{c_i}\right)\int_{\varepsilon_1=-\infty}^{+\infty}  f_\varepsilon(\varepsilon^+ | \varepsilon^0, \varepsilon_1) F_\varepsilon(\varepsilon_{1n}, \ldots)f_\varepsilon(\varepsilon_1) d\varepsilon_1,
\end{equation}
where $f_\varepsilon(\varepsilon_1)$ is the marginal distribution of $\varepsilon_1$, and
\[
F_\varepsilon(\varepsilon_{1n}, \ldots) = F_\varepsilon(\varepsilon_{1n}, 1, \ldots, 1, (V_{1n} -V_{in} + \varepsilon_{1n})_{i\in \C^0}).
\]

Equation \req{eq:density} corresponds to Equation~11 in \citeasnoun{BHAT2008274}.

If we assume that the $\varepsilon_{in}$ are independent, we obtain the MDCEV model introduced by \citeasnoun{BHAT2005679}. In that case,  the density \req{eq:density} is 
\begin{equation}
  f_e(e^+, e^0) = \mu^{J^+-1}\left(\prod_{i=1}^{J^+} c_i\right)\left(\sum_{i=1}^{J^+}\frac{1}{c_i}\right) \left( \frac{\prod_{i \in \C^+} e^{\mu V_{in}}}{(\sum_{i\in\C_n} e^{\mu V_{in}})^{J^+}} \right) (J^+-1)!,
\end{equation}
where the derivation is available in \citeasnoun{BHAT2008274}. Therefore, the contribution of observation $n$ to the log likelihood is
\begin{align*}
  \ln  f_e(e^+, e^0) =& (J^+-1) \ln \mu \\
  &+ \sum_{i=1}^{J^+} \ln c_i \\
  &+ \ln\left(\sum_{i=1}^{J^+}\frac{1}{c_i}\right)\\
  &+ \mu \sum_{i \in \C^+} V_{in} \\
  &- J^+ \ln \sum_{i\in\C_n} e^{\mu V_{in}} \\
  &+ \ln (J^+-1)!.
\end{align*}
Note that the last term is a constant, and is ignored by Biogeme.


\end{document}
